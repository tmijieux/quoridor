\section*{Introduction}

Dans le cadre de nos projets d'algorithme et de programmation 
impérative, nous disposions de 7 semaines pour développer un
jeu de Quoridor.\\

Le Quoridor est un jeu de société à deux ou quatres joueurs. Il a été inventé 
relativement récemment et fait partie des jeux pour lequels il n'existe pas 
d'algorithmes connus, capable de jouer une partie avec le niveau d'un joueur 
humain.

Le Quoridor est remarquable car il possède des règles simples, mais il présente 
une complexité assez conséquente.\\
Deux mesures classiques de la complexité d'un jeu sont:
\begin{itemize}
  \item les nombres d'états possibles du jeu,
  \item le nombre de parties différentes possibles.
\end{itemize}
Les ordres de grandeur pour le Quoridor sont respectivement de $10^{42}$ et de 
$10^{162}$, ce qui le place à une complexité comparable avec l échecs.  
\footnote{A Quoridor-playing Agent, p.3, Mertens}\\

Dans un premier temps, nous décrirons les besoins du projet en rappelant les 
règles du jeu. Ensuite, nous nous intéresserons à la conception du programme
que nous diviserons en deux parties : la partie \og Serveur \fg{} et la partie 
\og Stratégies \fg{}. Enfin, nous présenterons les résultats obtenus.
